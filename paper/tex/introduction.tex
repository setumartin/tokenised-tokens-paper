\section{Introduction}\label{sec:introduction}

We are witnessing a Cambrian explosion of tokens on blockchains:
Ethereum alone has hundreds of thousands of ERC-20 tokens.  Many
tokens are simple, in the sense that they are not composed of other
tokens.  But, some are.  For example, a liquidity pool token
represents a share of a collection of other tokens.  DEX
Screener~\cite{dex-screener-xx}, a popular liquidity pool tracker,
lists over one hundred thousand liquidity pool tokens.  Furthermore,
tokens are being composed in ever more creative ways:
\texttt{PT-weETH-25APR2024}~(\texttt{0xb18c87}) is a token issued by
Pendle Finance~\cite{nguyen-vuong-22} on the Arbitrum blockchain that
transitively depends on many other tokens.  At the time of writing,
this token has a market capitalisation of over one billion US dollars.
It is critical from the perspectives of technical and financial risk
to examine the composition of tokens.  If an investor purchases
\texttt{PT-weETH-25APR2024}, what tokens does the investment depend
upon?  In this paper, we present a novel method of examining token
composition at both the macro- and micro-level and we apply it to the
Ethereum blockchain.

% Antoher good example: Staked Curve.fi renBTC/wBTC Convex Deposit on
% Abra (\texttt{stkcvxcrvRenWBTC-abra})

Our method extracts \textit{meta-events} from EVM logs.  Low-level
events are emitted by contracts.  Meta-events are identified by
heuristics.  A single meta-event can be derived from multiple events.
For example, ERC-20 tokens (should) emit a \texttt{Transfer} event
whenever a token is transferred~\cite{vogelsteller-buterin-15}.  A
meta-event could signify a token being tokenised by another token,
i.e., a deposit of an underlying token with a contract and the minting
of a new share, or the burning of a share and the withdrawal of an
underlying token from a contract.  This meta-event, which we will call
a \textit{tokenising meta-event}, can be identified from multiple
\texttt{Transfer} events within a single transaction.  The tokenising
meta-events can be represented as a \textit{token graph}: each vertex
represents a token and each directed edge represents the token
corresponding to the source vertex being tokenised by the token
corresponding to the target vertex.  We apply various forms of graph
analysis to the token graph and we visualise the structure of token
compositions.

This paper is organised as follows.  In Sect.~\ref{sec:related-work}
we review related work.  In Sect.~\ref{sec:token-composition} we
introduce token composition, the token graph, and our data sources.
We present our analysis in Sect.~\ref{sec:analysis}.  Finally, we
conclude in Sect.~\ref{sec:conclusion}.
