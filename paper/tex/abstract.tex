Tokens have proliferated across blockchains in terms of number, market
capitalisation and utility.  Some tokens are tokenised versions of
existing tokens --- known variously as wrapped tokens, fractional
tokens, or shares.  The repeated application of this process creates
matryoshkian tokens of arbitrary depth.  We perform an empirical
analysis of token composition on the Ethereum blockchain.  We
introduce a graph that represents the tokenisation of tokens by other
tokens, and we show that the graph contains non-trivial topological
structure.  We relate properties of the graph, e.g., connected
components and cyclic structure, to the tokenisation process.  For
example, we identify the longest directed path and its corresponding
sequence of tokens, and we visualise the connected components relating
to a stablecoin and an NFT protocol.  Our goal is to explore and
visualise what has been wrought with tokens, rather than add yet
another brick to the edifice.
