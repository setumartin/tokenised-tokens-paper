\section{Related Work}\label{sec:related-work}

We categorise related work into four areas: empirical analysis of
smart contract composition and code reuse, the automatic detection of
tokens, graph analysis of blockchains and token systems, and wrapped
tokens.

Software composition is a hard problem~\cite{garlan-et-al-94}.  Smart
contracts on blockchains sidestep the low-level problems of
interoperability by using a shared execution environment (i.e., a
virtual machine) and de facto standards (e.g., ERCs), and the
high-level problems of architectural mismatch by taking a bottom-up
approach to composition.  He et al.~\cite{he-et-al-20} perform a
large-scale analysis of \num{10} million Ethereum smart contracts
deployed between July 2015 and December 2018.  They show that less
than \num{1}\% of the \num{10} million smart contracts are distinct,
and more than \num{63}\% of those are similar to at least one other
smart contract.  The results have been
replicated.~(\hspace{1sp}\cite{kondo-et-al-20,chen-et-al-21,khan-et-al-22})

Fr\"owis et al.~\cite{frowis-et-al-19} use symbolic execution to
automatically detect smart contracts on the Ethereum blockchain that
implement token functionality.  Di Angelo and
Salzer~\cite{di-angelo-salzer-21} reconstruct contract interfaces and
events from EVM bytecode.  Using primarily transaction data they
identify token contracts that comply with ERC standards and token
contracts that do not.  Oliveira et al.~\cite{oliveira-et-al-18}
propose a taxonomy for classifying tokens and they propose a decision
tree to guide the token design process.

Kitzler et al.~\cite{kitzler-et-al-21} select and analyse a subset of
DeFi activity on the Ethereum blockchain.  They construct and
topologically analyse two graphs: the \textit{contract account graph}
where the vertices are contract accounts and the edges are
transactions between those accounts, and the \textit{protocol graph}
where the vertices are protocols and the edges are transactions
between those protocols.  They show that community finding algorithms
identify communities in the contract account graph, but the
communities do not correspond to protocols.  There are several network
analyses of ERC-20 tokens on the Ethereum blockchain that quantify
their age, economic value, activity volume,
etc.~(\hspace{1sp}\cite{somin-et-al-18,victor-luders-19})

Caldarelli~\cite{caldarelli-21} discusses wrapped tokens and their
ability to represent real-world assets (tokenisation) and to bridge
tokens across blockchains (cross-blockchain interoperability).  The
WBTC whitepaper~\cite{kyber-et-al-xx} describes a general framework
for tokenising assets on a blockchain.  Santoro et
al.~\cite{santoro-et-al-22} propose a standard interface for vaults
for ERC-20 tokens.  The vaults can store a single \textit{asset} or
underlying token.  Users can \textit{deposit} or \textit{withdraw} the
asset.  In return, they receive \textit{shares} in the form of another
ERC-20 token.  Lloyd et al.~\cite{lloyd-et-al-23} analysed the
emergent outcomes of veTokens where a token is locked in exchange for
voting rights.

We use common terminology from graph theory through-out the paper.
Please refer to \cite{diestel-17} or a similar reference for
definitions.
