\section{Conclusion}\label{sec:conclusion}

In many fields, including engineering, chemistry, and cooking, the
complexity of a product arises from the combination of numerous base
materials or ingredients.  This principle holds true for tokens on
blockchains.  For example,
\texttt{stkcvxcrvRenWBTC-abra}~(\texttt{0xb65ede}) is a token that
represents a staked deposit of a share of a liquidity pool for
synthetic and wrapped versions of Bitcoin's native token.  The base
tokens (\texttt{renBTC}~(\texttt{0xeb4c27}),
\texttt{WBTC}~(\texttt{0x2260fa}), etc.) are combined to produce the
product.  In this paper we detail a novel graph representation of
token composition.  We construct the graph from the EVM logs of the
Ethereum blockchain and we relate its properties to the tokenisation
process.  For example, we highlight the role of stablecoins that can
be minted from various forms of collateral, and can be used as
collateral to mint other tokens.  In future work, we will refine the
heuristic for identifying tokenising meta-events to reduce the number
of false positives in the token graph.
